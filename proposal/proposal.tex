\documentclass[submission]{eptcs}
\setlength{\parindent}{0pt}
\providecommand{\event}{} % Name of the event you are submitting to

\title{Solid Construction from Surfaces}
\author{ Felipe Jara
\institute{Department of Computer Science\\
University of Chile\\
Santiago, Chile}
\email{fjararibet@gmail.com}
}
\begin{document}
\maketitle

\section{Introduction}

Techonology computer aided designed, commonly referred to as TCAD, is the electronic design automation that models integrated circuit fabrication and device operation [TCAD Book pag 1]. TCAD includes litography modeling, front end process modeling, device modeling, compact modeling, interconnect modeling, reliability modeling, equipment modeling, package simulation and materials modeling. [TCAD Book pag 2]

~Talk about TCAD implementation history and examples.

~Talk about ViennaPS and how it approches TCAD

~What ViennaPS does right and methods it uses

~What ViennaPS does wrong and what this thesis will talk about

~Why meshing is important


\section{Related work}


\section{Problem}

The representation of surfaces in ViennaPS is performed using the level set method, where the material interface is described as the zero level set of the signed distance function, which returns the shortest distance to the surface, defined over a regular grid. This representation results in a dense collection of surface points with no explicit connectivity, making direct meshing infeasible. While simple polygonization algorithms can generate meshes from these points, they often yield low-quality meshes due to the sheer number of redundant points and lack of surface structure. \\

To enable high-quality meshing, redundant points must be identified and removed without compromising the fidelity of the material interface. This requires developing a criterion for redundancy that ensures key geometric and topological properties of the original material M are preserved. These may include curvature, sharp features, connectivity, and multi-region boundaries. \\

The goal of this thesis is to develop and evaluate an algorithm that converts the level set surface output from ViennaPS into a high-quality polygon mesh, optimized for simulation and visualization purposes, while ensuring the recoverability of material properties from the simplified representation.

\section{Research questions}

\begin{itemize}
    \item Can an algorithm be developed to reconstruct multi-material polyhedral models from level set surfaces?
    \item Can these polyhedral models be simplified while preserving key material characteristics?
    \item Can high-quality meshes be generated from the simplified representations?
        \begin{itemize}
    \item What criteria can be used to identify and remove redundant points from level set surfaces without compromising geometry or material boundaries?
    \item How can the fidelity of the reconstructed mesh be measured with respect to the original level set representation?
\end{itemize}
\end{itemize}


\section{Hypothesis}

It is possible to develop an algorithm that reconstructs high-quality polygonal meshes from ViennaPS level set surfaces by removing redundant surface points based on geometric and topological criteria, such that the resulting mesh preserves the essential characteristics of the original multi-material structure.

\section{General Objective}

To develop an algorithm that generates high-quality polygon meshes from level set surfaces produced by ViennaPS, ensuring the preservation of key geometric and topological characteristics of the original multi-material structure.



\section{Specific Objectives}

\begin{itemize}
    \item To define geometric and topological criteria for identifying and removing redundant points from level set surfaces.
    \item To construct polyhedral surfaces from level set data, supporting multiple materials.
    \item To implement a simplification process that maintains essential surface features and material boundaries.
    \item To generate polygon meshes that are suitable for downstream applications such as simulation or visualization.
    \item To evaluate the quality and fidelity of the generated meshes with respect to the original level set representation.
\end{itemize}

\section{Methodology}

The methodology of this thesis consists of the following stages, each designed to address a specific objective and contribute to validating the proposed hypothesis:

\begin{enumerate}
    \item \textbf{Analysis of ViennaPS level set output} \\
    Study the structure and properties of the level set data generated by ViennaPS. Identify how surfaces are encoded (i.e., zero level set of the $\theta$ function) and extract surface points corresponding to material boundaries.

    \item \textbf{Design of simplification criteria} \\
    Define geometric and topological criteria for identifying redundant surface points. Geometric criteria may include local curvature, normal deviation, or proximity to neighbors, while topological criteria may involve preserving connected components and preventing surface disconnections.

    \item \textbf{Development of the reconstruction algorithm} \\
    Implement an algorithm that takes as input the extracted surface points and reconstructs a polygonal mesh. The algorithm should apply the simplification criteria to reduce point density while maintaining fidelity to the original surface.

    \item \textbf{Multi-material support} \\
    Extend the algorithm to distinguish and preserve boundaries between different materials. This may involve tagging level set surfaces or using multiple $\theta$ functions, ensuring that material interfaces are correctly reconstructed in the mesh.

    \item \textbf{Mesh generation and quality improvement} \\
    Use meshing techniques (e.g., Marching Cubes, Delaunay triangulation, or dual contouring) to generate the polygonal mesh. Post-process the mesh to improve quality — such as regularizing triangle shapes and smoothing without distorting features.

    \item \textbf{Evaluation and validation} \\
    Assess the performance and accuracy of the algorithm using both synthetic and real test cases. Evaluate the mesh based on:
    \begin{itemize}
        \item Fidelity to the original level set surface (e.g., Hausdorff distance)
        \item Mesh quality metrics (e.g., aspect ratio, valence uniformity)
        \item Preservation of material boundaries and topological features
    \end{itemize}
    Compare results against baseline methods or unprocessed level set outputs.
\end{enumerate}

\section{Expected results}

The following results are expected upon completion of this thesis:

\begin{itemize}
    \item A functional algorithm capable of reconstructing polygonal meshes from level set surfaces generated by ViennaPS.
    
    \item A set of well-defined geometric and topological criteria that can effectively identify and remove redundant surface points without compromising important features of the original structure.

    \item High-quality polygon meshes that:
    \begin{itemize}
        \item Accurately approximate the zero level set surface.
        \item Preserve the interfaces and boundaries between different materials.
        \item Maintain the topological integrity of the original model (e.g., connectivity and number of components).
    \end{itemize}
    
    \item Improved mesh quality compared to naïve or baseline meshing methods, as measured by quantitative metrics (e.g., triangle aspect ratio, Hausdorff distance, normal deviation).

    \item A general-purpose approach that can be integrated into ViennaPS workflows, enabling improved visualization, simulation, or further processing of semiconductor structures or other multi-material domains.
\end{itemize}


\nocite{*}
\bibliographystyle{eptcs}
\bibliography{generic}
\end{document}
