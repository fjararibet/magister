\documentclass[submission]{eptcs}
\setlength{\parindent}{0pt}
\providecommand{\event}{} % Name of the event you are submitting to

\title{Solid Construction from Surfaces}
\author{ Felipe Jara
\institute{Department of Computer Science\\
University of Chile\\
Santiago, Chile}
\email{fjararibet@gmail.com}
}
\begin{document}
\maketitle

\section{Introduction}

Techonology computer aided designed, commonly referred to as TCAD, is the electronic design automation that models integrated circuit fabrication and device operation [TCAD Book pag 1]. TCAD includes litography modeling, front end process modeling, device modeling, compact modeling, interconnect modeling, reliability modeling, equipment modeling, package simulation and materials modeling. [TCAD Book pag 2]

~Talk about TCAD implementation history and examples.

~Talk about ViennaPS and how it approches TCAD

~What Vienna does right and methods it uses

~What Vienna does wrong and what this thesis will talk about

~Why meshing is important


\section{Related work}



\subsection{ Estrategia  A}




\subsection{ Estrategia B}


\section{Problem}

The representation of surfaces in ViennaPS is performed using the level set method, where the material interface is described as the zero level set of the signed distance function, which returns the shortest distance to the surface, defined over a regular grid. This representation results in a dense collection of surface points with no explicit connectivity, making direct meshing infeasible. While simple polygonization algorithms can generate meshes from these points, they often yield low-quality meshes due to the sheer number of redundant points and lack of surface structure. \\

To enable high-quality meshing, redundant points must be identified and removed without compromising the fidelity of the material interface. This requires developing a criterion for redundancy that ensures key geometric and topological properties of the original material M are preserved. These may include curvature, sharp features, connectivity, and multi-region boundaries. \\

The goal of this thesis is to develop and evaluate an algorithm that converts the level set surface output from ViennaPS into a high-quality polygon mesh, optimized for simulation and visualization purposes, while ensuring the recoverability of material properties from the simplified representation.

\section{Research questions}

\begin{itemize}
    \item Can an algorithm be developed to reconstruct multi-material polyhedral models from level set surfaces?
    \item Can these polyhedral models be simplified while preserving key material characteristics?
    \item Can high-quality meshes be generated from the simplified representations?
        \begin{itemize}
    \item What criteria can be used to identify and remove redundant points from level set surfaces without compromising geometry or material boundaries?
    \item How can the fidelity of the reconstructed mesh be measured with respect to the original level set representation?
\end{itemize}
\end{itemize}


\section{Hypothesis}

It is possible to develop an algorithm that reconstructs high-quality polygonal meshes from ViennaPS level set surfaces by removing redundant surface points based on geometric and topological criteria, such that the resulting mesh preserves the essential characteristics of the original multi-material structure.

\section{General Objective}

To develop an algorithm that generates high-quality polygon meshes from level set surfaces produced by ViennaPS, ensuring the preservation of key geometric and topological characteristics of the original multi-material structure.



\section{Specific Objectives}

\begin{itemize}
    \item To define geometric and topological criteria for identifying and removing redundant points from level set surfaces.
    \item To construct polyhedral surfaces from level set data, supporting multiple materials.
    \item To implement a simplification process that maintains essential surface features and material boundaries.
    \item To generate polygon meshes that are suitable for downstream applications such as simulation or visualization.
    \item To evaluate the quality and fidelity of the generated meshes with respect to the original level set representation.
\end{itemize}

\section{Methodology}

\subsection{Método Bidimensional}
\begin{enumerate}
    \item Unir nube de puntos en polilínea.
    \item Cerrar polilínea usando el sólido original.
    \item Simplificar polígono sacando puntos colineales o casi colineales.
\end{enumerate}
\subsection{Método Trisimensional}

\section{Expected results}

 y contribuciones

\nocite{*}
\bibliographystyle{eptcs}
\bibliography{generic}
\end{document}
