\documentclass[submission]{eptcs}
\setlength{\parindent}{0pt}
\providecommand{\event}{} % Name of the event you are submitting to

\title{Solid Construction from Surfaces}
\author{ Felipe Jara
\institute{Department of Computer Science\\
University of Chile\\
Santiago, Chile}
\email{fjararibet@gmail.com}
}
\begin{document}
\maketitle

\section{Introduction}

Techonology computer aided designed, commonly referred to as TCAD, is the electronic design automation that models integrated circuit fabrication and device operation [TCAD Book pag 1]. TCAD includes litography modeling, front end process modeling, device modeling, compact modeling, interconnect modeling, reliability modeling, equipment modeling, package simulation and materials modeling. [TCAD Book pag 2]

~Talk about TCAD implementation history and examples.

~Talk about ViennaPS and how it approches TCAD

~What Vienna does right and methods it uses

~What Vienna does wrong and what this thesis will talk about

~Why meshing is important


\section{Related work}



\subsection{ Estrategia  A}




\subsection{ Estrategia B}


\section{Problem}

Since the representation of surfaces is done with the level set method with the \theta function defined in a grid, surfaces are represented with only points, which by themselves cannot be meshed. Simple algorithms can be used to generate a polygon with these points, but the level set representation makes it so there's a lot of points, so quality meshes can't be made. Points deemed redundant to describe the surface could be removed to make a better mesh, but a criteria needs to be decided to select redundant points and remove them without losing the material's characteristics, meaning we can retrieve the original material from the newly generated mesh.

[Bad quality mesh pic]
[Good quality mesh pic example]

The goal of this thesis is to develop an algorithm that takes the level set surface output of ViennaPS and produces a simplified, high-quality polygon mesh that preserves the material boundaries and characteristics of the original model M.
\section{Research questions}

\begin{itemize}
    \item ¿Se puede crear un algoritmo para reconstruir poliedros de múltiples materiales desde superficies?
    \item ¿Se puede simplificar los poliedros manteniendo características?
    \item ¿Se pueden generar mallas de calidad?
\end{itemize}


\section{Hypothesis}

Si se desarrolla un algoritmo que convierta una nube de puntos representando el borde de un material en un poliedro estructurado, entonces será posible generar una representación geométrica adecuada que conserve la forma y propiedades del material, permitiendo su manipulación y uso en simulaciones de microelectrónica.

\section{Main Goal}


\section{Specific goals}

\begin{itemize}
\item goal 1
\end{itemize}

\section{Methodology}

\subsection{Método Bidimensional}
\begin{enumerate}
    \item Unir nube de puntos en polilínea.
    \item Cerrar polilínea usando el sólido original.
    \item Simplificar polígono sacando puntos colineales o casi colineales.
\end{enumerate}
\subsection{Método Trisimensional}

\section{Expected results}

 y contribuciones

\nocite{*}
\bibliographystyle{eptcs}
\bibliography{generic}
\end{document}
